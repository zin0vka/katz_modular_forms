\documentclass{article}
\usepackage{amsmath,amsthm,amssymb,fullpage,enumerate,bm,hyperref}
\theoremstyle{definition}
\newtheorem{defi}{Definition}
\newtheorem{ex}{Exercise}
\title{Lecture 1\\Why didn't I learn about differential forms?}
\author{TM}
\date{\today}
\begin{document}
\section*{Introduction}
Friends, friends. I have a confession to make. I never actually taught myself any differential geometry. I'm a faker. My whole professional life is a lie. I should just stick to dumb as rocks programming and write {\tt /r/programmerhumor} memes.

But, I've never been one to stick to the sane option. So I'll do the insane thing -- teach myself differential geometry by way of giving lectures.
\section*{Motivation}
The thing is though. I have a reason to learn it. In order to actually give a full definition of modular forms in the geometrically most appropriate way, you really need to work with differential forms.

So what do I mean by the ``geometrically most appropriate way''? Well, I mean the way Nick Katz does it in his ground-breaking (or so I've been told) essay ``$p$-adic Properties of Modular Forms'' from 1972.

In order to begin to understand modular forms in his way, you need to start by viewing elliptic curves as ``tori''. Or to be more concrete, as quotients
\[\mathbb{C}/(\text{lattice in $\mathbb{C}$}).\]
What I mean by lattice in $\mathbb{C}$ here is an additive group of the form
\[\mathbb{Z}\omega_1+\mathbb{Z}\omega_2,\]
where $\omega_1$ and $\omega_2$ are points in the complex plane, not on the same line.

Now, an arbitrary point in the complex plane can be shifted in terms of integer multiples of $\omega_1$ and $\omega_2$ so that it sits inside the parallellogram spanned by $\omega_1$ and $\omega_2$. Take a look at the graphic below.
\begin{figure}[h]
  \centering
  {\tt Insert TikZ graphic here.}
\end{figure}

What this means is that $\mathbb{C}/\text{lattice}$ is topologically the same as a parallellogram with sides identified in an orientation-preserving way.

\begin{figure}[h]
  \centering
  {\tt Insert TikZ graphic here.}
\end{figure}

From topology we know this is homeomorphic to a torus. Hence, certainly $\mathbb{C}/\text{lattice}$ must be a $2$-dimensional manifold.

Is it though? (Answer: yes, of course.)
\section*{What's Spivak say?}
Spivak starts by defining a {\tt diffeomorphism}.
\begin{defi}
  Let $U,V\subseteq\mathbb{R}^n$ are open (here and henceforth in the Euclidean topology). Then a differentiable function $h:U\to V$ (here and henceforth, a $C^\infty$ function) with a differentiable inverse $h^{-1}:V\to U$ is called a {\tt diffeomorphism}.
\end{defi}
From this ey (Spivak pronoun) defines a $k$-dimensional manifold.
\begin{defi}
  Let $M\subset\mathbb{R}^n$ and $k$ a non-negative integer. Then $M$ is called a $k$-dimensional manifold if for every point $x\in M$ it holds that
  \begin{quotation}
    There exists an open set $\mathbb{R}^n\supseteq U\ni x$ and an open set $V\subseteq\mathbb{R}^n$ and a diffeomorphism $h:U\to V$ satisfying
    \[h(U\cap M)=V\cap(\mathbb{R}^k\times\{\mathbf{0}\}).\]
  \end{quotation}
\end{defi}
Spivak tells us to think of this $U\cap M$ being up to diffeomorphism the same as $\mathbb{R}^k\times\{\mathbf{0}\}$.

For funsies, let's show that $\mathbb{T}^2$ indeed is a $2$-manifold.
\begin{ex}
  Show that $\mathbb{T}^2$ is a $2$-manifold.
\end{ex}
\begin{proof}[Solution]
  First we need to settle on which definition of $\mathbb{T}^2$ we want to use. Since we want it to sit in $\mathbb{R}^n$ for some $n$, we should probably use the good ole ``donut''-definition, which makes it sit in $\mathbb{R}^3$.

  We define it parametrically:
  \[\begin{cases}
    x(\theta,\phi)=(R+r\cos\theta)\cos\phi\\
    y(\theta,\phi)=(R+r\cos\theta)\sin\phi\\
    z(\theta,\phi)=r\sin\theta.
  \end{cases}\]
  Where $0\leq\theta,\phi<2\pi$ and $R$ is ``the distance from the center of the tube to the center of the torus'' and $r$ is ``the radius of the tube''. Obviously the particular choice of $R$ and $r$ is irrelevant.

  In other words
  \[\mathbb{T}^2=\{(x(\theta,\phi),y(\theta,\phi),z(\theta,\phi)):(\theta,\phi)\in [0,2\pi)^2\}.\]

  \begin{figure}[h]
    \centering
    {\tt TikZ me!}
  \end{figure}
  
  Let now $(x,y,z)\in\mathbb{T}^2$ be arbitrary.
\end{proof}
\end{document}
