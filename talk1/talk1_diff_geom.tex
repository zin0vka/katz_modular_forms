\def\pgfsysdriver{pgfsys-dvipdfm.def}
%Bug in xelatex, see https://github.com/josephwright/beamer/issues/337
\documentclass[usenames,dvipsnames]{beamer}
\usepackage{tikz}
\usepackage{amsmath,amsthm,amssymb,enumerate,bbm,hyperref,mathtools,xeCJK,xpinyin,tikz-cd,stmaryrd,adjustbox,lscape,listings,changepage,epigraph,xcolor}
\usepackage[linesnumbered,ruled]{algorithm2e}
\usecolortheme{crane}

\usepackage{pgfpages}
\setbeamertemplate{note page}[plain]
\setbeameroption{show notes on second screen=right}

\usepackage{tkz-euclide,tkz-fct}

\newcommand\dhrightarrow{%
  \mathrel{\ooalign{$\rightarrow$\cr%
  $\mkern3.5mu\rightarrow$}}
}

\newcommand\dhxrightarrow[2][]{%
  \mathrel{\ooalign{$\xrightarrow[#1\mkern4mu]{#2\mkern4mu}$\cr%
  \hidewidth$\rightarrow\mkern4mu$}}
}

\newcommand\Wider[2][3em]{%
\makebox[\linewidth][c]{%
  \begin{minipage}{\dimexpr\textwidth+#1\relax}
  \raggedright#2
  \end{minipage}%
  }%
}


\theoremstyle{definition}
\newtheorem{defi}{Definition}
\newtheorem{theo}{Theorem}

\title{Elliptic curves, lattices, and some differential geometry}
\subtitle{A ``modular forms''-y gumbo}
\author{Tobias Magnusson}
\institute{Chalmers University of Technology}
\date{\today}
\subject{Mathematics}
\begin{document}
\begin{frame}[plain]
\titlepage
  \note{Thanks Step! So, in my first talk of {\tt TISHK}, I'll talk about something that has been bugging me for ages -- Katz' ``geometric'' definition of modular forms.

  I have been told by an expert of the field that Katz' point of view is {\bf the} way to view modular forms, mainly for the ease of changing your ground field. When doing computations, this will just be the field that your Fourier coefficients belong to.

  There are other ways of changing the ground field (Serre does something cool for example), but from cursorily reading Katz, I can see what it is appealing to many folk. It seems ``simple'' in a way, not as ``ad-hoc''-y as what Serre does.

  The only problem is that it's just ``simple'' if you know all of the involved prerequisites. That's where I'm stumped.}
\end{frame}

\begin{frame}
  \frametitle{Outline}
  \begin{itemize}
    \item Classical definition of modular forms.
    \item Katz' definition.
    \item Weierstraß' $\mathcal{P}$-function -- elliptic curves are tori.
    \item Tobias is stumped 1: differential form.
    \item A concrete example!
    \item Visualize: {\tt pyqtgraph}.\footnote{Including an interesting proramming problem -- intersecting {\tt MeshData}.}
    \item Tobias is stumped 2: lattices of periods.
    \item Next talk -- the Tate curve.
  \end{itemize}
  \note{In this talk}
\end{frame}
\end{document}
